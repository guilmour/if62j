\documentclass[12pt, a4paper]{article}	
\usepackage[utf8]{inputenc}
\usepackage{graphicx}
\usepackage{wrapfig}
\usepackage{lipsum}
\usepackage{geometry}
\usepackage{amsfonts}
\newcommand{\Z}{\mathbb{Z}}
\newlength{\tabcont}
\usepackage{amsmath}
\usepackage{amssymb}
\usepackage[ampersand]{easylist}
\renewcommand{\refname}{Referências}
\usepackage{hyperref}
\renewcommand{\figurename}{Fig.}


\newcommand{\linespacing}{1.5}
\renewcommand{\baselinestretch}{\linespacing}

\geometry{a4paper,left=3cm,right=2cm,bottom=3.5cm,top=2cm, headsep=1cm, footskip=1.5cm}

\title{Resenha: Epistomologia da Física Clássica}
\author{Guilmour Rossi}
\date{7 de Setembro, 2015}



\begin{document}     

\begin{minipage}{20cm}
\begin{wrapfigure}{l}{2.5cm}
\includegraphics[width=2.5cm]{ex_scientia_artis.png}  
\end{wrapfigure}

\textbf{\\Universidade Tecnológica Federal do Paraná\\
		Departamento Acadêmico de Informática \\
    	IF62J - Oficina de Integração I \\
    	Prof. Luiz Nacamura Junior \\}

\end{minipage}   
    
       \begin{center}             
        Tarefa Prática Avaliativa \texttt{\#}{1} 
		\textbf{{\\Guilmour Rossi \\}}
        {Março, 2016.\\}		        
   \end{center}  
   


\section{Objetivo}

Analisar tecnicamente artigos científicos relacionados a arduino disponíveis em canais de busca voltados ao meio acadêmico.

\section{Análise}
\subsection{``Física com Arduino para iniciantes"
}
	
Marisa A. Cavalcante, Cristiane Rodrigues Caetano Tavolaro e Elio Molisani, Revista Brasileira de Ensino de Física v.33, n.4, 4503, (2011). Disponível em \url{http://www.scielo.br/pdf/rbef/v33n4/18.pdf} acesso em 16/03/2016.

    \vskip10mm
    
Este artigo publicado na Revista Brasileira de Ensino de Física em 2011 traz informações valiosas para o ensino da física usando a plataforma de prototipação em código-aberto \textit{arduino}. O texto começa analisando como o ensino diferenciado e mais prático da física pode auxiliar o aprendizado de estudantes na rede de ensino, usando de várias referências de trabalhos e pesquisas sobre o assunto. Com uma linguagem simples e direta, os autores ainda fornecem exemplificações e demostrações do como fazer diversos experimentos, unindo-os com a teoria que pode ser proposta em sala de aula. As equações e demonstrações gráficas da parte teórica por trás de cada projeto pode aturdir um pouco o estudante de ensino médio interessado que por ventura faça uma análise superficial do texto, mas basta uma leitura mais atenta para perceber a facilidade como o assunto é tratado, fazendo jus ao título do artigo. O artigo ainda conta com várias figuras úteis ao aprendizado, que ilustram com clareza aquilo que deve ser feito; sendo algumas imagens feitas pelos próprios autores e outras obtidas pela internet, com sua respectiva menção aos autores e fonte. O artigo ainda possui três apêndices com o código-fonte dos projetos em \textit{arduino}, possibilitando ao estudante, além do aprendizado e a visualização dos conceitos físicos propostos, implementar e estudar a programação dos projetos, podendo expandi-los e aprimorá-los de acordo com sua necessidade. O artigo é recomendado para todos os estudantes e educadores que gostem tanto de física como de eletrônica, sendo muito rico em informações para que possamos abrilhantar e aprimorar a forma de ensino em nossas escolas. 




%\bibliographystyle{abnt}
%\bibliography{bib.bib}








\end{document}